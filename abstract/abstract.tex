
%\phantomsection
\addcontentsline{toc}{chapter}{Abstract}

\begin{abstract}

%Reconfigurable system is a computer architecture combining the flexibility of software with the performance of hardware by processing with programmable computing fabric.
%In particular, \glspl{fpga} is the underlying technology which consist of prefabricated logic and routing resources.
%Reconfigurable systems are commonly used for prototyping hardware designs and acceleration of software applications.
This thesis proposes novel approaches to design and optimise reconfigurable systems targeting real-time applications.

%The performance gains in \glspl{fpga} are obtained by maximising data and task parallelism, but resource limitation of \gls{fpga} often restricts the level of parallelism.
Our first contribution in this thesis is to propose novel data structures and memory architectures for accelerating real-time proximity queries, with potential application to robotic surgery. We optimise performance while maintaining accuracy by several techniques including mixed precision, function transformation and streaming data flow. Significant speedup is achieved using our reconfigurable accelerator platforms over double-precision CPU, GPU and FPGA designs.

%As \glspl{fpga} are increasingly being deployed for \gls{hpc} applications, power dissipation of \glspl{fpga} is a concern. 
The second contribution of this thesis is an adaptation methodology for real-time sequential Monte Carlo methods. Based on workload over time, different configurations with various performance and power consumption tradeoffs are loaded onto the FPGAs dynamically. Promising energy reduction has been achieved in addition to speedup over CPU and GPU designs. The approach is evaluated in an application to robot localisation.

%Productivity and design time have long been the challenges to bring \glspl{fpga} to more wide-spread usage. 
The third contribution to this thesis is a design flow for automated mapping and optimisation of real-time sequential Monte Carlo methods. Machine learning algorithms are used to search for an optimal parameter set to produce the highest solution quality while satisfying all timing and resource constraints. The approach is evaluated in an application to air traffic management.

\end{abstract}
